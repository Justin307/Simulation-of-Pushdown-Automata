\chapter{Závěr}\label{chap:Conclusion}

Cílem této práce bylo vytvořit aplikaci, která by uživateli umožňovala graficky simulovat činnost zásobníkových automatů. Jako první bylo nutné si ale nastudovat problematiku zásobníkových automatů, jak jsou definovány, jaké jsou jejich typy a jak fungují. Poté jsem vytvořil webovou aplikaci, dovoluje uživateli simulovat činnost libovolného zásobníkového automatu pro jím zadaný vstup. Zásobníkové automaty může uživatel nahrát jako soubor nebo je vytvořit přímo v aplikaci. Všechny zásobníkové automaty se ukládají do lokálního úložiště prohlížeče, aby k nim měl uživatel přístup a při příštím spuštění aplikace. Následně jsem vytvořil několik vzorových zásobníkových automatů, na kterých jde vidět činnost aplikace a pomocí kterých jsem aplikaci testoval.

Práci je možné v budoucnu rozšířit a další funkce, jako je např.\ možnost převodu mezi automaty přijímajícími prázdným zásobníkem a přijímajícími stavy, grafické zobrazení automatu nebo možnost vytvoření zásobníkového automatu z bezkontextové gramatiky.
\endinput