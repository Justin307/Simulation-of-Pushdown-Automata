\chapter{Závěr}\label{chap:Conclusion}

Cílem této práce bylo vytvořit aplikaci, která by uživateli umožňovala graficky simulovat činnost zásobníkových automatů. Jako první bylo nutné si ale nastudovat problematiku zásobníkových automatů, jak jsou definovány, jaké jsou jejich typy a jak fungují. Poté byla vytvořena webová aplikace, dovolující uživateli simulovat činnost libovolného zásobníkového automatu pro jím zadaný vstup. Zásobníkové automaty může uživatel nahrát jako soubor nebo je vytvořit přímo v aplikaci. Všechny zásobníkové automaty se ukládají do lokálního úložiště prohlížeče, aby k nim měl uživatel přístup i při příštím spuštění aplikace. Následně bylo vytvořeno několik vzorových zásobníkových automatů, na kterých jde vidět činnost aplikace a pomocí kterých byla aplikace testována.

Práci je možné v budoucnu rozšířit o další funkce, jako je např.\ možnost převodu mezi automatem přijímajícím prázdným zásobníkem a přijímajícími stavy, grafické zobrazení automatu nebo možnost vytvoření zásobníkového automatu z bezkontextové gramatiky.
\endinput