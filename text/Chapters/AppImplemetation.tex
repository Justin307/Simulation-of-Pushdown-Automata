\chapter{Implementace aplikace}

Na předchozích stránkách teto práce byly popsány zásobníkové automaty, specifikace aplikace a technologie, které v této práci používám. Následující stránky se budou zabývat již samotnou implementací aplikace. Nejprve se zaměřím na to, jak vůbec reprezentovat zásobníkový automat v kódu. Následovat pak bude část zaměřující se na samotný simulátor a v poslední části se zaměřím na menu, tvorbu automatů pomocí formuláře, nahrávání souborů a jejich ukládání do paměti.

\section{Reprezentace zásobníkových automatů v kódu}

Abych mohl se zásobníkovými automaty pracovat v kódu, musel jsem mít způsob, jak je reprezentovat v kódu. Vytvořil jsem si tedy třídu PushdownAutomata, viz kód~\ref{src:PushdownAutomataDefinition}. Tato třída obsahuje jako atributy jednotlivé části definice zásobníkových automatů. 

\begin{lstlisting}[label=src:PushdownAutomataDefinition, caption={Deklarce třídy PushdownAutomata}]
    class PushdownAutomata{
        states: State[];
        inputSymbols: InputSymbol[];
        stackSymbols: StackSymbol[];
        initialState: State;
        initialStackSymbol: StackSymbol;
        acceptingState: State[] | null;
        transitionFunction: TransitionFunction[];
    
        checkInputTapeValidity(inputTape: string): string[];
        getTransitionFunctions(tapeSymbol: string, state: State, stackSymbol:  StackSymbol | null): TransitionFunction[];
    }
\end{lstlisting}

První tři atributy definují jednotlivé množiny symbolů a stavů, se kterými automat pracuje. Jsou pro ně vytvořeny nové datové typy, zdrojový kód~\ref{src:PushdownAutomataTypes}. Všechny tyto typy obsahují atribut value, který obsahuje samotnou hodnotu. Typ InputSymbol navíc obsahuje ještě atribut isEpsilon, který je využíván u přechodových funkcí a umožňuje přechod bez přečtení symbolu ze vstupu.

\begin{lstlisting}[label=src:PushdownAutomataTypes, caption={Datové typ State, StackSymbol, InputSymbol}]
    type State = {
        value: string;
    }
    type StackSymbol = {
        value: string;
    }
    type InputSymbol = {
        isEpsilon: boolean;
        value?: string;
    }
\end{lstlisting}

Dále následují dva atributy definující výchozí konfiguraci automatu --- initialState a initialStackSymbol. Po nich následuj acceptingState, který může nabývat dvou různých hodnot. Pokud obsahuje hodnotu null, tak zásobníkové automat přijímá slovo prázdným zásobníkem. V opačném případě, kdy obsahuje pole stavů, je slovo přijímáno přijímacím stavem.

Posledním atributem je transitionFunction. Ten obsahuje pole všech přechodových funkcí, které jsou reprezentované opět svým typem TransitionFunction, zdrojový kód~\ref{src:TransitionFunctionType}. Ten se skládá z 5 atributů --- počátečního stavu, symbolu na zásobníku, symbolu na vstupní pásce, nového stavu a množiny zásobníkových symbolů, které budou přidány na zásobník, v tomto pořadí.

\begin{lstlisting}[label=src:TransitionFunctionType, caption={Datové typ TransitionFunction}]
    type TransitionFunction = {
        fromState: State;
        startSymbol: StackSymbol;
        inputSymbol: InputSymbol;
        toState: State;
        pushedSymbols: StackSymbol[];
    }
\end{lstlisting}


\endinput