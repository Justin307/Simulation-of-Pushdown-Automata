\chapter{Úvod}    
Chomského hierarchie popisuje 4 druhy gramatik a jazyků --- regulární, bezkontextové, kontextové a neomezené. Pokud pracujeme s regulárními jazyky, nejnižšími z těchto 4 tříd, tak nám pro výpočet stačí konečné automaty, ať už deterministické nebo nedeterministické. Pokud bychom ale chtěli pracovat s bezkontextovými jazyky, tak by nám konečný automat nestačil. Pro bezkontextové jazyky tedy musíme použít zásobníkový automat, který má oproti konečným automatům navíc zásobník pro ukládaní dat. Právě zásobníkovými automaty se táto práce zabývá, přesněji simulací jejich činností.

Cílem této práce bylo implementovat grafický simulátor zásobníkových automatů, deterministických i nedeterministických, přijímajících prázdným zásobníkem i přijímacími stavy. Aplikace by měla sloužit studentům nebo komukoliv, kdo se zajímá o zásobníkové automaty, a měla by jím umožnit jednodušeji pochopit, jak v nich probíhá výpočet.

Aplikace umožňuje:
\begin{itemize}
    \item Zadat definici automatu přímo v aplikaci
    \item Nahrát automat ze souboru
    \item Stáhnout automat jako souboru
    \item Upravit automat
    \item Provést nad automatem simulaci pro uživatelem zadaný vstup
\end{itemize}

\section{Obsah práce}

%TODO: Check chapter order
Práce je rozdělená do několika kapitol a jejich obsah je popsán dále. Kapitola~\ref{chap:PushdownAutomata} se zabývá tím, co to jsou zásobníkové automaty, jak jsou definovány, rozdíly mezi typy zásobníkových automatů --- deterministické a nedeterministické, přijímající prázdným zásobníkem a přijímacím stavem a jak probíhá výpočet. V kapitole~\ref{chap:ExistingApps} jsou zmíněny některé již existující aplikace řešící stejné téma. Kapitola~\ref{chap:AppSpecifications} se pak věnuje tomu, co se od aplikace očekává a jaké jsou požadavky. Kapitola~\ref{chap:AppImplemetation} obsahuje popis technologií, které budou využity a samotnou implementaci funkcionalit. Následující kapitola~\ref{chap:UI} se pak zabývá uživatelským rozhraním a jeho ovládáním. Kapitola~\ref{chap:Inputs} obsahuje popis vstupních souborů obsahujících definice zásobníkových automatů a v poslední kapitole~\ref{chap:Conclusion} je nakonec shrnutá celá práce, její výsledek a možná rozšíření.
\endinput