\chapter{Úvod}    
    Chomského hierarchie popisuje 4 druhy gramatik a jazyků --- regulární, bezkontextové, kontextové a neomezené. Pokud pracujeme s regulárními jazyky, nejnižší z těchto 4 tříd, tak nám pro výpočet stačí konečné automaty, ať už deterministické nebo nedeterministické. Pokud bychom ale chtěli pracovat s bezkontextovými jazyky, tak by nám konečný automat nestačil. Pro bezkontextové jazyky tedy musíme použít zásobníkový automat, který má oproti konečným automatům navíc zásobník pro ukládaní dat. Právě zásobníkovými automaty se táto práce zabývá, přesněji simulací jejich činností.

    Cílem této práce je implementovat grafický simulátor zásobníkových automatů, deterministických i nedeterministických, přijímajících prázdným zásobníkem i přijímacími stavy. 
    
    Aplikace bude umožňovat:

    \begin{itemize}
        \item Zadat definici automatu přímo v aplikaci
        \item Nahrát automat ze souboru
        \item Stáhnout automat jako souboru
        \item Upravit automat
        \item Provést nad automatem simulaci pro uživatelem zadaný vstup
    \end{itemize}

    Práce bude rozdělená do několika kapitol. Kapitola~\ref{chap:PushdownAutomata} se bude zabývat tím, co to jsou zásobníkové automaty, jak jsou definovány, rozdíly mezi typy zásobníkových automatů --- deterministické a nedeterministické, přijímající prázdným zásobníkem a přijímacím stavem a jak probíhá výpočet. Kapitola~\ref{chap:AppSpecifications} se pak bude věnovat tomu, co se od aplikace očekává a jaké jsou požadavky. Kapitola~\ref{chap:AppImplemetation} bude obsahovat popis technologií, které budou využity a samotnou implementaci funkcionalit. Následující kapitola~\ref{chap:UI} se pak bude zabývat uživatelským rozhraním a jeho ovládáním. Kapitola~\ref{chap:Inputs} bude obsahovat popis vstupních souborů obsahujících definice zásobníkových automatů. V předposlední kapitol7~\ref{chap:ExistingApps} budou zmíněny některé již existující aplikace řešící stejné téma a v poslední kapitole~\ref{chap:Conclusion} bude nakonec shrnutá celá práce a její výsledek.
\endinput